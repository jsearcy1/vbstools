\section{Introduction}
The discovery of a Higgs-like boson at the LHC~\cite{ATLAS_higgs,CMS_higgs} was the first step toward a better understanding of the  
%The discovery of a Higgs-like boson at the LHC~\cite{ATLAS_higgs,CMS_higgs} was the first step toward determining the properties of 
electroweak symmetry breaking (EWSB) mechanism. One important, and unverified, prediction of the Standard Model (SM) 
is that the scattering amplitude of longitudinal vector bosons ($V_{L}V_{L} \rightarrow V_{L}V_{L}$) is unitarized by the Higgs boson. 
Measuring VBS processes at a hadron collider, however, is experimentally challenging. Both ATLAS and CMS collaborations recently provided 
the first evidence of a VBS process using events with two same-sign $W$ boson in association with two forward jets ($pp \rightarrow$ \ssWW)~\cite{ATLAS_ssWW,CMS_ssWW}. 
This final state has the advantage of relatively small SM background contributions compared with other VBS processes. 
While an ideal candidate for first observing the VBS process, measuring the longitudinal fraction of these events is not straight forward. 

In general the polarization of a gauge boson can be determined from the angular distribution of its decay products. 
Since a $W$ boson only couples to left-handed particles and right-handed anti-particles. In the rest frame of the $W$ boson, the decayed charged lepton 
is expected to preferentially pointing in the $W$ boson spin direction for $W^+$ and in the opposite $W$ boson spin direction for $W^-$.  
The normalized differential cross section of a leptonically-decaying $W$ boson can be written in terms of polarization fractions as
\begin{multline}
 \frac{1}{\sigma} \frac{d\sigma}{d\cts} = \frac{3}{8} f_- (1 \mp \cts)^2 + \frac{3}{8} f_+ (1 \pm \cts)^2 \\ 
+\frac{3}{4} f_L (1 - \ctsSq) , {\text {for~}} W^\pm 
\end{multline}
%  R. Ellis, W. Stirling, and B. Webber, QCD and collider physics
%, Camb. Monogr. Part. Phys. Nucl. Phys. Cosmol. 8 (1996) 1
where \ts is the decay angle defined  in the $W$'s rest frame as the angle between the charged lepton and the $W$ boson direction of motion. 
The three fraction parameters $f_{-}$, $f_{+}$ and $f_L$ denote the fractions of $W$ events with three possible helicity states $-1$, $+1$ and 0, respectively. 
The $f$s are subjected to the constraint $f_- + f_+ + f_L = 1$. 
To measure \ts, we need to fully reconstruct the direction of motion and also the center-of-mass of the $W$ boson. 

To determine the charge of each $W$ boson, we require both bosons decay leptonically in the $pp \rightarrow$ \ssWW 
events which results in two neutrinos in the final state. Since neutrinos escape the detection, the $W$ boson's rest frames cannot be directly measured. 
It is thus difficult to determine the polarization fractions of each boson and the fraction of longitudinal scattering events in the \ssWW process.

Many proposals have been made to determine longitudinal fractions in other VBS final states, such as semileptonic $WW$~\cite{Han:2009em}, $WZ$~\cite{aa} and $ZZ$ 
(le houche report) or fully-leptonic $WZ$ and $ZZ$, where full event kinematics could be reconstructed. However, these channels either suffer from large SM backgrounds 
not present in the \ssWW channel or have relatively low production cross sections. Attempts have been made to gain sensitivity through other variables than \ts in the \ssWW 
channel. One example is the variable $R_{p_T}=(p_{T}^{\ell 1} \times p_{T}^{\ell 2}) / (p_T^{j_1} \times p_T^{j_2})$ shown in ~\cite{Doroba:2012pd}, where $\ell_1$ 
and $\ell_2$ denote the two leptons in no particular order and $j_1$ and $j_2$ denote the two most energetic jets in the event. 
It is natural to assume that all of the sensitivity to longitudinal scattering is not encompassed in a single variable, and that better discrimination could be obtained by 
combining all the event information with a machine learning technique. In this paper we develope a method to use a neural network to map measurable quantities to 
the truth \cts values that contain the polarization information of the two $W$ bosons. 

\section{Machine learning model}
While it has become common practice in high energy physics to use multi-variate techniques to separate signal 
from background, to the author's knowledge multi-variate regression has not been used to directly predict underlying quantities. 
Recent success with deep learning in other areas of HEP~\cite{Baldi:2014pta,Baldi:2014pta}. 

For the \ssWW events, we use representative measurable quantities such as the transverse momentum ($p_T$), pseudorapidity ($\eta$) and 
azimuthal angle ($\theta$) of two leptons and two jets, and $x$ and $y$ components of the missing transverse energy ($\slashed{E}_T^x$ and $\slashed{E}_T^y$). 
The overall number of measurable quantities we used is 14. The goal of the multi-variate technique is thus to find the best mapping from these measurable quantities 
to the two truth values of \cts (one for each $W$ boson) present in each event. 
We choose a multi-layer neural network with a final output layer with linear activation. The neural network was implemented with the 
Theano software packages~\cite{theano}. Hyperparameters were tuned by hand, but undoubtedly could be improved. 
The cost function is defined as the mean error squared as listed below
\small
\begin{equation}
{\cal{C}}= \sum_{i=1}^{N} [(\ctsnb_{1,i} -N_{1,i})^2+(\ctsnb_{2,i}-N_{2,i})^2]/2N
\end{equation}
\normalsize
where $N$ is the overall number of events used, $\ctsnb_{1/2,i}$ is the truth value of \cts for each $W$ boson with random ordering for the $i$-th event, 
and $N_{1,2/i}$ is the value of the two neural network outputs. 
Stochastic gradient descent algorithm~\cite{sgd} is used to find all weights and biases in the neural network that minimizes the overall cost function using a training sample. 

We generate *** \ssWW events using the {\sc madgraph} event generator~\cite{madgraph} at a proton-proton center-of-mass energy of 13 TeV. 
The NN23l01 parton distribution function~\cite{pdf} is used as the default. These events are then 
passed through the detector response simulation of the ATLAS detector implemented in {\sc delphes}~\cite{delphes}. 
Events are split into three categorizes: 1/4 are used in a training sample, 1/4 are used as a validation test against over training, and the remaining 1/2 
are used to build templates and do sensitivities studies. A ** layer neural network with ** hidden neurons and a learning rate of ** is used.

%\begin{figure}
%\includegraphics[width=.45\textwidth]{./fig/OneD_cts_compare.pdf}
%\caption{\label{fig:mlp_err} Distance of prediction from the true value of ($\ctsnb(1)$,$\ctsnb(2)$)}
%\end{figure}

\section{\label{sec:signal} Signal Model}
To reduce the contributions from other SM processes, we apply the following generator level cuts on the \ssWW sample: 
\begin{itemize}
\item Parton $p_T > 20$ GeV and $|\eta| < 5$;
\item Lepton $p_T > 10$ GeV and $|\eta| < 2.5$;
\item $\dR_{jj}>0.4$, $\dR_{\ell \ell}>0.4$ and $\dR_{\ell, j} > 0.4$;
\item Invariant mass of the two partons $M_{jj}> 150$ GeV.
\end{itemize}
The resulting cross section at $13$ TeV is 8.4 fb$^{-1}$ which is used to normalized the expected number of signal events.

We will first demonstrate the usefulness of deep learning networks with this general sample and then discuss the effects of cuts 
to reject other backgrounds as well as detector modeling effects. 

Polarization fractions are first obtained on the default sample by fitting the two-dimensional distribution of the truth \cts variable for each $W$ boson. 
In order to fit for these polarization fractions templates must be built for ``pure'' polarization states. These templates are created by reweighting each event based on the truth \cts distribution. The weight $W_i$ for the $i$-th event is given by $W_i = F_i/n_i$, where $n_i$ is used for the normalization and is defined as 
\begin{multline}
n_i=[ \frac{3}{8} f_- (1 \mp \ctsnb_1)^2 + \frac{3}{8} f_+ (1 \pm \ctsnb_1)^ 2 +\frac{3}{4} f_0 (1 - \ctsSqnb_1)] \\
\times [ \frac{3}{8} f_- (1 \mp \ctsnb_2)^2 + \frac{3}{8} f_+ (1 \pm \ctsnb_2)^ 2 +\frac{3}{4} f_0 (1 - \ctsSqnb_2)]. 
\end{multline}
$F_i$ represents the six possible polarization states for the two $W$ bosons: Left-Left ($--$), Left-Right ($-+$), Right-Right($++$), 
Left-Longitudinal ($-L$), Right-Longitudinal ($+L$), or Longitudinal-Longitudinal ($LL$). They are defined as 
\small
\begin{equation}
F_i \in  \left( \begin{array}{c} 
  --=f_-^2 (1 \mp \ctsnb_1)^2(1 \mp \ctsnb_2)^2,\\
  -+=f_- f_+[ (1 \mp \ctsnb_1)^2(1 \pm \ctsnb_2)^2\\ \;\;+(1 \pm \ctsnb_1)^2 (1 \mp \ctsnb_2)^2],\\
  ++=f_+^2 (1 \pm \ctsnb_1)^2(1 \pm \ctsnb_2)^2,\\
  -L=f_- f_L[ (1 \mp \ctsnb_1)^2(1 - \ctsSqnb_2)\\  \;\;+(1 - \ctsSqnb_1))(1 \mp \ctsnb_2)^2 ],\\
  +L=f_+ f_L[ (1 \pm \ctsnb_1)^2 (1 - \ctsSqnb_2))\\  \;\;+(1 - \ctsSqnb_1)(1 \pm \ctsnb_2)^2 ],\\
  LL=f_L^2 (1 - \ctsSqnb_1))(1 - \ctsSqnb_2)\\
\end{array} \right).
\end{equation}
\normalsize
Since no ordering is applied to the two bosons we require that the individual polarization fractions $f_-$,$f_+$,$f_L$ are the same for both $W$ bosons. 
For reweighting the original sample $f_-$, $f_+$, $f_L$  are take as a function of the invariant mass of the diboson system ($M_{WW}$). 
Weights are calculated before any additional event level cuts are made, and the resulting templates are remade for each set of cuts explored. 
To validate the reweighting procedure, we also generate pure polarization state samples using {\sc madspin} and compare the obtained events kinematics 
with those obtained from the reweighted sample. Figure~\ref{fig:2D_polarization_six} shows the calculated $\cts_1$ vs $\cts_2$ distribution for all six polarization states. 

\begin{figure}
\includegraphics[width=.49\textwidth]{./fig/2D_temps.pdf}
\caption{\label{fig:2D_polarization_six}Two dimensional calculated $\cts_1$ vs $\cts_2$ templates after reweighting to pure polarization states which clock-wise from the upper left LO,OO,RO,LL,LR,RR.}
\end{figure}
 
While there are six possible polarization combinations, not all polarization states are as interesting from the the stand point of new physics, 
and better measurements can be preformed if assumptions are made. 
We combine events with both $W$ bosons transversly-polarized as $TT$ which is the sum of $--$, $-+$ and $++$ combinations, events with one $W$ boson transversly-polarized and one $W$ boson longitudinally-polarized as $TL$ which is the sum of $-L$ and $+L$ combinations, and events with both $W$ bosons longitudinally-polarized 
as $LL$. Figure~\ref{fig:2D_polarization_three} shows the calculated $\cts_1$ vs $\cts_2$ distribution for these three polarization states. 

\begin{figure}
\includegraphics[width=.49\textwidth]{./fig/templates.pdf}
\caption{\label{fig:2D_polarization_three}Two dimensional calculated $\cts_1$ vs $\cts_2$ templates after reweighting to three polarization states: $TT$ (left), $TL$ (middle) and $LL$ (right).}
\end{figure}

Figure~\ref{fig:Rpt} shows the normalized $R_{p_T}$ templates for these three polarization states. The differences at large $R_{p_T}$ indicate the sensitivity 
of this variable to diferent polarization states.
\begin{figure}
\includegraphics[width=.45\textwidth]{./fig/templates_2D.pdf}
\caption{\label{fig:Rpt}$R_{p_T}$ templates for three polarization states: $TT$, $TL$ and $LL$.}
\end{figure}

\section{Results}

Armed with templates for each polarization state and a distribution that is sensitive to different polarization states,  
all we have to do is fit the resulting calculated $\cts_1$ vs $\cts_2$ two-dimensional distribution in data to derive each polarization fraction. 
In actual data analysis, this would involve first selecting events to remove SM backgrounds and then subtracting predicted background from data. 
The effect of additional selection criteria and backgrounds is covered below, but for the case of this section we assume that 
{\sc madgraph} selection can be fitted directly to obtain sensitivities. 
The maximum likelihood fit is preformed with the RooFit framework~\cite{aa}. Fit uncertainties are determined by 
randomly fluctuating data expectations within their Poisson errors and repeating the fit, and confidence intervals are derived from the toy experiments. 
To validate the fitting method, we also make sure the fitted values for each fraction agree with with the values obtained at the truth level.  

The precisions for all six polarization fractions as a function of the integrated luminosity assumed are listed in Fig.~\ref{fig:sen_6}. 
The corresponding precisions by fitting the $R_{p_T}$ distributions are also shown. Better precisions are obtained for the calculated $\cts$ 
variables which indicate that it is a more sensitive variable to different polarization states than $R_{p_T}$.  
The precision for the LL fraction is **\% (**\%) for an integrated luminosity of 100 (3000) fb$^{-1}$.

The precisions for all three polarization fractions as a function of the integrated luminosity assumed are listed in Fig.~\ref{fig:sen_3}. 
The corresponding precisions by fitting the $R_{p_T}$ distributions are also shown. 
It can be seen that as expected better precision can be obtained by reducing the number of templates used. 
Transverse components can be measured with great precision, whereas separating pure longitudinal-longitudinal scattering 
from longitudinal-transverse scattering is challenging. However, if we fix the TL fraction to the SM prediction, we 
can perform a precise measurement of the LL fraction, as shown in Fig.~\ref{fig:sen_2}. In all cases the neural network output outperforms the 
$R_{pT}$ variable.

\begin{figure}
\includegraphics[width=.45\textwidth]{./fig/sens_0.pdf}
\caption{\label{fig:sen_6} 68\% and 95\% confidence intervals for fits with six templates as a function of the integrated luminosity assumed. 
Fits of the neural network on the left and fits of $R_{pT}$ the on the right.}
\end{figure}

\begin{figure}
\includegraphics[width=.45\textwidth]{./fig/sens_1.pdf}
\caption{ \label{fig:sen_3} 68\% and 95\% confidence intervals for fits with three templates as a function of the integrated luminosity assumed. Fits of the neural network on the left and fits of $R_{pT}$ on the right.}
\end{figure}

\begin{figure}
\includegraphics[width=.45\textwidth]{./fig/sens_2.pdf}
\caption{\label{fig:sen_2} 68\% and 95\% confidence intervals for fits with two templates as a function of the integrated luminosity assumed. Fits of the neural network on the left and fits of $R_{pT}$ on the right.}
\end{figure}

While the success of this neural network at the parton level is encouraging, it is important to check if this procedure will stand
up to experimental realities of finite detector resolution and non-VBS background. To reduce SM backgrounds in the loose fiducial region as defined in 
Sect.~\ref{sec:signal} , we apply similar cuts as used by the ATLAS collaboration~\cite{ATLAS_ssWW} to obtain a tight fiducial region:

\begin{itemize}
\item Jet $p_T > 30$ GeV;
\item Lepton $p_T > 25$ GeV;
\item $\slashed{E}_T > 40$ GeV;
\item $M_{jj} > $500 GeV;
\item $|\Delta Y_{jj}| > 2.4 $.
\end{itemize}

After the application of these cuts the dominate background comes from the $WZ$ production where one lepton is not detected or not reconstructed.  
For the generated \ssWW events, we also use {\sc pythia6} for parton shower and hadronization simulation and then pass these events through the 
{\sc delphes} package (using the CMS simulation card)~\cite{delphes} to simulate the detector response of a general-purpose particle detector at the LHC. 
We define the loose and tight fiducial regions at both the parton level and also at the {\sc delphes} level. 
Table~\ref{tab:sens} shows the extracted limits on TT, TL and LL fractions for loose and tight fiducial regions defined 
at the parton level and at the {\sc delphes} level. 
Table~\ref{tab:sens2} shows the corresponding numbers on TT$+$TL and LL fractions. 
An integrated luminosity of ** fb$^{-1}$ is assumed. 

Figure~\ref{sig_bkg} shows the calculated $\cts$ distribution for the dominant $WZ$ background, $++$ and $LL$ components of the \ssWW events. 
It can be seen that the calculated $\cts$ distribution for the $WZ$ background closely resembles that of the $++$ template, 
it will be important to subtract the $WZ$ contribution before the actual fitting. 

\begin{table}[h!]
\begin{center}
\begin{tabular}{c |cc|cc|cc}
\hline \hline
Fiducial region & \multicolumn{2}{c|}{TT} & \multicolumn{2}{c|}{TL}  & \multicolumn{2}{c}{LL} \\
\hline
 & LL & UL & LL & UL & LL & UL  \\
\hline
Loose (parton level) & 0.51 & 0.62 &0.27 & 0.48 &0.01 & 0.13\\ 
Tight (parton level) & 0.47 & 0.63 &0.19 & 0.52 &0.0 & 0.19\\ 
Loose ({\sc delphes} level) & 0.48 & 0.65 &0.16 & 0.52 &0.0 & 0.21\\ 
Tight ({\sc delphes} level) & 0.44 & 0.66 &0.08 & 0.56 &0.0 & 0.27\\ 
\hline
\hline
\end{tabular}
\caption{\label{tab:sens} Lower Limits (LL) and Upper Limits (UL) on the TT, TL, and LL fractions for loose and tight fiducial regions defined at the parton level and at the {\sc delphes} level. }
\end{center}
\end{table}

\begin{table}[h!]
\begin{tabular}{c |cc|cc}
\hline \hline
Fiducial region & \multicolumn{2}{c|}{TT$+$TL} & \multicolumn{2}{c}{LL} \\
\hline
 & LL & UL & LL & UL  \\
\hline
Loose (parton level) & 0.92 & 0.95 &0.05 & 0.09\\ 
Tight (parton level) & 0.89 & 0.95 &0.05 & 0.11\\ 
Loose ({\sc delphes} level) & 0.9 & 0.95 &0.05 & 0.11\\ 
Tight ({\sc delphes} level) & 0.87 & 0.95 &0.05 & 0.13\\ 
\hline
\hline
\end{tabular}
\caption{\label{tab:sens2}Lower Limits (LL) and Upper Limits (UL) on the TT$+$TL and LL fractions for loose and tight fiducial regions defined at the parton level and at the {\sc delphes} level.}
\end{table}

\begin{figure}
\includegraphics[width=.45\textwidth]{./fig/sig_bkg.pdf}
\caption{\label{sig_bkg} Signal and background shapes for the calculated $\cts$.}

\end{figure}

\section{Conclusions}
We present a method to determine the fraction of each polarization state in the \ssWW events by using deep machine learning technique. 
This method allows us to recover the charged lepton angular distributions from measurable event kinematics. 
We compare the results obtained from this method and from other traditional methods and show the advantage of our method. 
Cuts to reject backgrounds as well as detector smearing reduces the sensitivity as expected, but the method remains a useful tool
for the study of VBS.

\begin{thebibliography}{99}
%\cite{Chatrchyan:2012ufa}
%\cite{Aad:2012tfa}
\bibitem{ATLAS_higgs} 
  G.~Aad {\it et al.}  [ATLAS Collaboration],
  %``Observation of a new particle in the search for the Standard Model Higgs boson with the ATLAS detector at the LHC,''
  Phys.\ Lett.\ B {\bf 716}, 1 (2012)
  [arXiv:1207.7214 [hep-ex]].
  %%CITATION = ARXIV:1207.7214;%%
  %4423 citations counted in INSPIRE as of 02 Jun 2015


\bibitem{CMS_higgs} 
  S.~Chatrchyan {\it et al.}  [CMS Collaboration],
  %``Observation of a new boson at a mass of 125 GeV with the CMS experiment at the LHC,''
  Phys.\ Lett.\ B {\bf 716}, 30 (2012)
  [arXiv:1207.7235 [hep-ex]].
  %%CITATION = ARXIV:1207.7235;%%
  %4343 citations counted in INSPIRE as of 02 juin 2015

%\cite{Aad:2014zda}
\bibitem{ATLAS_ssWW} 
  G.~Aad {\it et al.}  [ATLAS Collaboration],
  %``Evidence for Electroweak Production of $W^{\pm}W^{\pm}jj$ in $pp$ Collisions at $\sqrt{s}=8$ TeV with the ATLAS Detector,''
  Phys.\ Rev.\ Lett.\  {\bf 113}, no. 14, 141803 (2014)
  [arXiv:1405.6241 [hep-ex]].
  %%CITATION = ARXIV:1405.6241;%%
  %26 citations counted in INSPIRE as of 02 juin 2015


%\cite{Khachatryan:2014sta}
\bibitem{CMS_ssWW} 
  V.~Khachatryan {\it et al.}  [CMS Collaboration],
  %``Study of vector boson scattering and search for new physics in events with two same-sign leptons and two jets,''
  Phys.\ Rev.\ Lett.\  {\bf 114}, no. 5, 051801 (2015)
  [arXiv:1410.6315 [hep-ex]].
  %%CITATION = ARXIV:1410.6315;%%
  %6 citations counted in INSPIRE as of 02 juin 2015


%\cite{Han:2009em}
\bibitem{Han:2009em} 
  T.~Han, D.~Krohn, L.~T.~Wang and W.~Zhu,
  %``New Physics Signals in Longitudinal Gauge Boson Scattering at the LHC,''
  JHEP {\bf 1003}, 082 (2010)
  [arXiv:0911.3656 [hep-ph]].
  %%CITATION = ARXIV:0911.3656;%%
  %38 citations counted in INSPIRE as of 02 juin 2015

%\cite{Doroba:2012pd}
\bibitem{Doroba:2012pd} 
  K.~Doroba, J.~Kalinowski, J.~Kuczmarski, S.~Pokorski, J.~Rosiek, M.~Szleper and S.~Tkaczyk,
  %``The $W_L W_L$ Scattering at the LHC: Improving the Selection Criteria,''
  Phys.\ Rev.\ D {\bf 86}, 036011 (2012)
  [arXiv:1201.2768 [hep-ph]].
  %%CITATION = ARXIV:1201.2768;%%
  %11 citations counted in INSPIRE as of 02 juin 2015

%\cite{Baldi:2014kfa}
\bibitem{Baldi:2014kfa} 
  P.~Baldi, P.~Sadowski and D.~Whiteson,
  %``Searching for Exotic Particles in High-Energy Physics with Deep Learning,''
  Nature Commun.\  {\bf 5}, 4308 (2014)
  [arXiv:1402.4735 [hep-ph]].
  %%CITATION = ARXIV:1402.4735;%%
  %2 citations counted in INSPIRE as of 02 juin 2015

%\cite{Baldi:2014pta}
\bibitem{Baldi:2014pta} 
  P.~Baldi, P.~Sadowski and D.~Whiteson,
  %``Enhanced Higgs Boson to $\tau^+\tau^-$ Search with Deep Learning,''
  Phys.\ Rev.\ Lett.\  {\bf 114}, no. 11, 111801 (2015)
  [arXiv:1410.3469 [hep-ph]].
  %%CITATION = ARXIV:1410.3469;%%

\end{thebibliography}

